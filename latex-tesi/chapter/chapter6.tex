% !TEX encoding = UTF-8
% !TEX TS-program = pdflatex
% !TEX root = ../main.tex
% !TEX spellcheck = en-EN

%************************************************


The goal of this thesis work was to use neural networks as an alternative to point methods that have been done in the past. So studying the state of the art on the use
of the various solutions developed has allowed us to understand what was the current state of the use of networks in this area.

Subsequently, we studied how neural networks adapted to the different datasets taken into analysis. And finally we analyzed the results obtained comparing them at first
with the reference dataset and then with the komatsuna. The better performances obtained with SOLOv2 also identified a weakness in the network such as the adaptation
to unfamiliar contexts and therefore the difficulty in finding leaf instances.

We can conclude that our method succeeds in dealing effectively with individual plants and that it manages to coincidentally identify both covered and uncovered leaves. 

In the future we can imagine several possible developments both that of an improvement of the networks in order to improve even more the finding of leaf instances but
also that of predictive models able to develop on the basis of the shape and depth images a model able to predict the real size of the leaf, which is difficult at this
time given the visual limitation of still images.



